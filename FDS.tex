\documentclass[12pt,openany]{book}


% Load packages
\usepackage[utf8]{inputenc} % for unicode input
\usepackage{geometry} % for page layout
\usepackage{graphicx} % for including images
\usepackage{hyperref} % for hyperlinks
\usepackage{tocbibind} % includes the bibliography in the table of contents
\usepackage{amsmath} % for advanced math formatting
\usepackage{amsfonts} % for mathematical fonts
\usepackage{amssymb} % for mathematical symbols
\usepackage{lipsum} % generates filler text
\usepackage{fancyhdr}
\usepackage{amsthm} % for theorem environments
\usepackage{amsmath}
\usepackage[table]{xcolor} % for cell coloring
\usepackage{colortbl} % for cell coloring
\usepackage{graphicx} % for scaling the table
\usepackage{booktabs} % For professional looking tables
\usepackage[table]{xcolor} % For cell coloring
\usepackage[normalem]{ulem} % For underlining

% Document geometry (page size, margins)
\geometry{a4paper, left=25mm, right=25mm, top=30mm, bottom=30mm}
% Custom page style for centered page numbers
\pagestyle{fancy}
\fancyhf{} % clear all header and footer fields
\fancyfoot[C]{\thepage} % center page number
\renewcommand{\headrulewidth}{0pt}
\renewcommand{\footrulewidth}{0pt}



% Document begins
\begin{document}


% Title Page
\begin{titlepage}
    \centering
    \vspace*{1cm}
    
    
    \vspace{0.5cm}
    \Huge
    Fundamentals of digital systems
    \vspace{0.5cm}
    
    \LARGE
    Ali EL AZDI
    
    \vfill
    
    A description or preface here
    
    February 19, 2024

    
\end{titlepage}

\cleardoublepage
% Table of Contents
\tableofcontents

% List of Figures (Uncomment if needed)
% \listoffigures

% List of Tables (Uncomment if needed)
% \listoftables

\mainmatter% Chapter 1
\chapter{Number Systems}


% Section 1.1
\section{Digital Representations}
\text{In digital representation, a number is represented by an ordered n-tuple (a Digit-vector string) : }
\begin{itemize}
    \item[] In a digital representation, a number is represented by an ordered n-tuple:
    \begin{itemize}
        \item[] The n-tuple is called a \textbf{digit vector}, each element is a \textbf{digit}
        \item[] The number of digits $n$ is called the precision of the representation. (careful! leftward indexing)
        
    
    \end{itemize}
\end{itemize}
\[ X  = (X _{n-1}, X _{n-1}, \ldots, X _{0}) \]
\begin{itemize}
    \item[] Each digit is given \textbf{a set of values} $D_{i}$ 
    (eg. For base 10 representation of numbers, $D_i = \{0,1,2, \dots, 9\}$)
    \item[] the \textbf{Set size}, the maximum number of representable digit vectors  is : $K = \prod_{i=0}^{n-1} |D_{i}|$
\end{itemize}



  
  




\subsection{(Non)Redundant Number Systems}
     A number system is nonredundant if each digit-vector represents a different integer

\subsection{Weighted Number Systems}
The rule of representation if a Weighted (Positional) Number Systems is as follows :
$\;\displaystyle\sum_{i=0}^{n-1} X_{i}W_{i}$ where $W = (W_{n-1}, W_{n-2}, \dots, W_{0})$

\subsection{Radix Systems}
When weights are in this format : 
\[
\begin{cases}
   $$W_{0} = 1 \\
    W_{i+1} = W_{i}R_{i} \;\; \text{with } 1 \leq i \leq n-1$$
    \end{cases} \]





Also written :
$W_{0} = 1, \; \prod_{j=0}^{i-1} R_{j}$

\subsection{Fixed and Mixed-Radix Number Systems}
\begin{itemize}
    \item[] In a \textbf{fixed-radix system}, all elements of the radix-vector have the same value \( r \) (\textit{the radix})
    \item[] The weight vector in a fixed-radix system is given by:
    \begin{equation*}
      W = \left( r^{n-1}, r^{n-2}, \ldots, r^2, r, 1 \right)
    \end{equation*}
    and the integer \( x \) becomes:
    \begin{equation*}
      x = \sum_{i=0}^{n-1} X_i \times r^i
    \end{equation*}
    \item[] In a \textbf{mixed-radix system}, the elements of the radix-vector differ
  \end{itemize}
  \subsection*{Example: Decimal Number System}
The decimal number system has the following characteristics:
\begin{itemize}
    \item Radix \(r = 10\), indicating it is a fixed-radix system.
\end{itemize}
The weight vector \(W\) is defined as:
\begin{equation*}
    W = \left(10^{n-1}, 10^{n-2}, \ldots, 10^2, 10, 1\right)
\end{equation*}
An integer \(x\) in this system is represented by:
\begin{equation*}
    x = \sum_{i=0}^{n-1} X_i \times 10^i
\end{equation*}
For example:
\begin{equation*}
    854703 = 8 \times 10^5 + 5 \times 10^4 + 4 \times 10^3 + 7 \times 10^2 + 0 \times 10^1 + 3 \times 10^0
\end{equation*}

In contrast, the representation can be fixed or mixed:
\begin{description}
    \item[Fixed:] This category includes:
    \begin{itemize}
        \item Decimal – radix 10
        \item Binary – radix 2
        \item Octal – radix 8
        \item Hexadecimal – radix 16
    \end{itemize}
    \item[Mixed:] An example of mixed representation is the way time is represented in terms of hours, minutes, and seconds:
    \begin{itemize}
        \item Radix-vector \(R = (24, 60, 60)\)
        \item Weight-vector \(W = (3600, 60, 1)\)
    \end{itemize}
\end{description}

\subsection{Canonical Number Systems}

\begin{itemize}
  \item[] In a \textbf{canonical number system}, the set of values for a digit \( D_i \) is with \( |D_i| = R_i \), the corresponding element of the radix vector
  \begin{equation*}
    D_i = \{0, 1, \ldots, R_i - 1\}
  \end{equation*}
  \item[] Canonical digit sets with fixed radix:
  \begin{itemize}
    \item Decimal: \{0, 1, \ldots, 9\}
    \item Binary: \{0, 1\}
    \item Hexadecimal: \{0, 1, 2, \ldots, 15\}
  \end{itemize}
  \item[] Range of values of \( x \) represented with \( n \) fixed-radix-\( r \) digits:
  \begin{equation*}
    0 \leq x \leq r^n - 1
  \end{equation*}
  A system with fixed positive radix r and a canonical set of digit values is called a radix-r conventional number system.
\end{itemize}
\newpage
\section{Binary/Octal/Hexadecimal to/from Decimal}
\section*{Conversion Table}
\begin{itemize}
  \item[] The hexadecimal system supplements 0-9 digits with the letters A-F.
  \vspace{-5px}
  \item[] \textit{Remark.} Programming languages often use the prefix 0x to denote a hexadecimal number.
\end{itemize}

% Define the color for the table cells
\definecolor{cellblue}{rgb}{0.88, 0.94, 0.97}
\definecolor{cellpink}{rgb}{0.96, 0.88, 0.89}
\definecolor{cellgreen}{rgb}{0.906, 1, 0.831}
\vspace{-20px}

\begin{table}[ht]
    \centering
    \caption{Conversion table up to 15.}
    \vspace{10px}
    \label{my-label}
    \begin{tabular}{@{}cccc@{}}
    \toprule
    \textbf{Decimal} & \textbf{Binary} & \textbf{Octal} & \textbf{Hexadecimal} \\ \midrule
     0 & \cellcolor{cellgreen}0000 & \cellcolor{cellblue}00 & \cellcolor{cellpink}0 \\
     1 & \cellcolor{cellgreen}0001 & \cellcolor{cellblue}01 & \cellcolor{cellpink}1 \\
     2 & \cellcolor{cellgreen}0010 & \cellcolor{cellblue}02 & \cellcolor{cellpink}2 \\
     3 & \cellcolor{cellgreen}0011 & \cellcolor{cellblue}03 & \cellcolor{cellpink}3 \\
     4 & \cellcolor{cellgreen}0100 & \cellcolor{cellblue}04 & \cellcolor{cellpink}4 \\
     5 & \cellcolor{cellgreen}0101 & \cellcolor{cellblue}05 & \cellcolor{cellpink}5 \\
     6 & \cellcolor{cellgreen}0110 & \cellcolor{cellblue}06 & \cellcolor{cellpink}6 \\
     7 & \cellcolor{cellgreen}0111 & \cellcolor{cellblue}07 & \cellcolor{cellpink}7 \\
     8 & \cellcolor{cellgreen}1000 & \cellcolor{cellblue}10 & \cellcolor{cellpink}8 \\
     9 & \cellcolor{cellgreen}1001 & \cellcolor{cellblue}11 & \cellcolor{cellpink}9 \\
    10 & \cellcolor{cellgreen}1010 & \cellcolor{cellblue}12 & \cellcolor{cellpink}A \\
    11 & \cellcolor{cellgreen}1011 & \cellcolor{cellblue}13 & \cellcolor{cellpink}B \\
    12 & \cellcolor{cellgreen}1100 & \cellcolor{cellblue}14 & \cellcolor{cellpink}C \\
    13 & \cellcolor{cellgreen}1101 & \cellcolor{cellblue}15 & \cellcolor{cellpink}D \\
    14 & \cellcolor{cellgreen}1110 & \cellcolor{cellblue}16 & \cellcolor{cellpink}E \\
    15 & \cellcolor{cellgreen}1111 & \cellcolor{cellblue}17 & \cellcolor{cellpink}F \\\bottomrule
    \end{tabular}
    \end{table}
    
    To convert a binary number to hexadecimal, group the binary digits in sets of four, starting from the right. Each group of four binary digits corresponds to one hexadecimal digit.

    \textbf{Example:}
    \begin{align*}
    \text{Binary:} & \quad \underline{1010}\,\underline{1100} \\
    \text{Hexadecimal:} & \quad A\,C \quad (\text{Since } \underline{1010}_2 = A_{16} \text{ and } \underline{1100}_2 = C_{16})
    \end{align*}
    
    To convert a binary number to octal, group the binary digits in sets of three, also starting from the right. Each group of three binary digits corresponds to one octal digit.
    
    \textbf{Example:}
    \begin{align*}
    \text{Binary:} & \quad \,\,\,\underline{100}\,\underline{101}\,\underline{110} \\
    \text{Octal:} & \quad 4\,5\,6 \quad (\text{Since } \underline{100}_2 = 4_8, \underline{101}_2 = 5_8, \text{ and } \underline{110}_2 = 6_8)
    \end{align*}



% Chapter 2
\chapter{Literature Review}
\lipsum[4-5]

% Section 2.1
\section{Previous Work}
\lipsum[6]

% Add more chapters as needed

\backmatter% Bibliography
% Replace 'sample.bib' with your actual bibliography file name
% \bibliographystyle{plain}
% \bibliography{sample}

\end{document}

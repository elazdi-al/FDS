\documentclass[12pt,openany, tikz,border=10pt]{book}


% Load packages
\usepackage[utf8]{inputenc} % for unicode input
\usepackage{geometry} % for page layout
\usepackage{graphicx} % for including images
\usepackage{hyperref} % for hyperlinks
\usepackage{tocbibind} % includes the bibliography in the table of contents
\usepackage{amsmath} % for advanced math formatting
\usepackage{amsfonts} % for mathematical fonts
\usepackage{amssymb} % for mathematical symbols
\usepackage{lipsum} % generates filler text
\usepackage{fancyhdr}
\usepackage{amsthm} % for theorem environments
\usepackage{amsmath}
\usepackage[table]{xcolor} % for cell coloring
\usepackage{colortbl} % for cell coloring
\usepackage{graphicx} % for scaling the table
\usepackage{booktabs} % For professional looking tables
\usepackage[table]{xcolor} % For cell coloring
\usepackage[normalem]{ulem} % For underlining
\usepackage[document]{ragged2e}
\usepackage{tikz}
\usetikzlibrary{arrows.meta,calc,decorations.markings,math,arrows.meta}

% Document geometry (page size, margins)
\geometry{a4paper, left=25mm, right=25mm, top=30mm, bottom=30mm}
% Custom page style for centered page numbers
\pagestyle{fancy}
\fancyhf{} % clear all header and footer fields
\fancyfoot[C]{\thepage} % center page number
\renewcommand{\headrulewidth}{0pt}
\renewcommand{\footrulewidth}{0pt}



% Document begins
\begin{document}


% Title Page
\begin{titlepage}
    \centering
    \vspace*{1cm}
    
    
    \vspace{0.5cm}
    \Huge
    Fundamentals of digital systems
    \vspace{0.5cm}
    
    \LARGE
    Ali EL AZDI
    
    \vfill
    
    A description or preface here
    
    February 19, 2024

    
\end{titlepage}

\cleardoublepage
% Table of Contents
\tableofcontents

% List of Figures (Uncomment if needed)
% \listoffigures

% List of Tables (Uncomment if needed)
% \listoftables

\mainmatter% Chapter 1
\chapter{Number Systems}


% Section 1.1
\section{Digital Representations}
\begin{itemize}
    \item[] In a digital representation, a number is represented by an ordered n-tuple:
    \begin{itemize}
        \item[] The n-tuple is called a \textbf{digit vector}, each element is a \textbf{digit}
        \item[] The number of digits $n$ is called the precision of the representation. (careful! leftward indexing)
        
    
    \end{itemize}
\end{itemize}
\[ X  = (X _{n-1}, X _{n-1}, \ldots, X _{0}) \]
\begin{itemize}
    \item[] Each digit is given \textbf{a set of values} $D_{i}$ 
    (eg. For base 10 representation of numbers, $D_i = \{0,1,2, \dots, 9\}$)
    \item[] The \textbf{set size}, the maximum number of representable digit vectors is: $K = \prod_{i=0}^{n-1} |D_{i}|$
\end{itemize}



  
  




\subsection{(Non)Redundant Number Systems}
     A number system is nonredundant if each digit-vector represents a different integer

\subsection{Weighted Number Systems}
The rule of representation if a Weighted (Positional) Number Systems is as follows :
$$\;\displaystyle\sum_{i=0}^{n-1} X_{i}W_{i}$$ where $$W = (W_{n-1}, W_{n-2}, \dots, W_{0})$$

\subsection{Radix Systems}
When weights are in this format : 
\[
\begin{cases}
   $$W_{0} = 1 \\
    W_{i+1} = W_{i}R_{i} \;\; \text{with } 1 \leq i \leq n-1$$
    \end{cases} \]





Also written :
$W_{0} = 1, \; \prod_{j=0}^{i-1} R_{j}$

\subsection{Fixed and Mixed-Radix Number Systems}
\begin{itemize}
    \item[] In a \textbf{fixed-radix system}, all elements of the radix-vector have the same value \( r \) (\textit{the radix})
    \item[] The weight vector in a fixed-radix system is given by:
    \begin{equation*}
      W = \left( r^{n-1}, r^{n-2}, \ldots, r^2, r, 1 \right)
    \end{equation*}
    and the integer \( x \) becomes:
    \begin{equation*}
      x = \sum_{i=0}^{n-1} X_i \times r^i
    \end{equation*}
    \item[] In a \textbf{mixed-radix system}, the elements of the radix-vector differ
  \end{itemize}
  \subsection*{Example: Decimal Number System}
The decimal number system has the following characteristics:
\begin{itemize}
    \item Radix \(r = 10\), it's a fixed-radix system.
\end{itemize}
The weight vector \(W\) is defined as:
\begin{equation*}
    W = \left(10^{n-1}, 10^{n-2}, \ldots, 10^2, 10, 1\right)
\end{equation*}
An integer \(x\) in this system is represented by:
\begin{equation*}
    x = \sum_{i=0}^{n-1} X_i \times 10^i
\end{equation*}
For example:
\begin{equation*}
    854703 = 8 \times 10^5 + 5 \times 10^4 + 4 \times 10^3 + 7 \times 10^2 + 0 \times 10^1 + 3 \times 10^0
\end{equation*}

\subsubsection{Examples of Fixed and Mixed radix systems}
\begin{description}
  \item[Fixed:] The base of number systems.
  \begin{itemize}
      \item Decimal – radix 10
      \item Binary – radix 2
      \item Octal – radix 8
      \item Hexadecimal – radix 16
  \end{itemize}
  \item[Mixed:] An example of a mixed radix representation, such as time:
  \begin{itemize}
      \item Radix-vector \(R = (24, 60, 60)\)
      \item Weight-vector \(W = (3600, 60, 1)\)
  \end{itemize}
\end{description}

\subsection{Canonical Number Systems}

\begin{itemize}
  \item[] In a \textbf{canonical number system}, the set of values for a digit \( D_i \) is with \( |D_i| = R_i \), the corresponding element of the radix vector
  \begin{equation*}
    D_i = \{0, 1, \ldots, R_i - 1\}
  \end{equation*}
  \item[] Canonical digit sets with fixed radix:
  \begin{itemize}
    \item Decimal: \{0, 1, \ldots, 9\}
    \item Binary: \{0, 1\}
    \item Hexadecimal: \{0, 1, 2, \ldots, 15\}
  \end{itemize}
  \item[] Range of values of \( x \) represented with \( n \) fixed-radix-\( r \) digits:
  \begin{equation*}
    0 \leq x \leq r^n - 1
  \end{equation*}
  A system with fixed positive radix r and a canonical set of digit values is called a radix-r conventional number system.
\end{itemize}
\newpage
\section{Binary/Octal/Hexadecimal to/from Decimal}
\section*{Conversion Table}
\begin{itemize}
  \item[] The hexadecimal system supplements 0-9 digits with the letters A-F.
  \vspace{-5px}
  \item[] \textit{Remark.} Programming languages often use the prefix 0x to denote a hexadecimal number.
\end{itemize}

% Define the color for the table cells
\definecolor{cellblue}{rgb}{0.88, 0.94, 0.97}
\definecolor{cellpink}{rgb}{0.96, 0.88, 0.89}
\definecolor{cellgreen}{rgb}{0.906, 1, 0.831}
\vspace{-20px}

\begin{table}[ht]
    \centering
    \caption{Conversion table up to 15.}
    \vspace{10px}
    \label{my-label}
    \begin{tabular}{@{}cccc@{}}
    \toprule
    \textbf{Decimal} & \textbf{Binary} & \textbf{Octal} & \textbf{Hexadecimal} \\ \midrule
     0 & \cellcolor{cellgreen}0000 & \cellcolor{cellblue}00 & \cellcolor{cellpink}0 \\
     1 & \cellcolor{cellgreen}0001 & \cellcolor{cellblue}01 & \cellcolor{cellpink}1 \\
     2 & \cellcolor{cellgreen}0010 & \cellcolor{cellblue}02 & \cellcolor{cellpink}2 \\
     3 & \cellcolor{cellgreen}0011 & \cellcolor{cellblue}03 & \cellcolor{cellpink}3 \\
     4 & \cellcolor{cellgreen}0100 & \cellcolor{cellblue}04 & \cellcolor{cellpink}4 \\
     5 & \cellcolor{cellgreen}0101 & \cellcolor{cellblue}05 & \cellcolor{cellpink}5 \\
     6 & \cellcolor{cellgreen}0110 & \cellcolor{cellblue}06 & \cellcolor{cellpink}6 \\
     7 & \cellcolor{cellgreen}0111 & \cellcolor{cellblue}07 & \cellcolor{cellpink}7 \\
     8 & \cellcolor{cellgreen}1000 & \cellcolor{cellblue}10 & \cellcolor{cellpink}8 \\
     9 & \cellcolor{cellgreen}1001 & \cellcolor{cellblue}11 & \cellcolor{cellpink}9 \\
    10 & \cellcolor{cellgreen}1010 & \cellcolor{cellblue}12 & \cellcolor{cellpink}A \\
    11 & \cellcolor{cellgreen}1011 & \cellcolor{cellblue}13 & \cellcolor{cellpink}B \\
    12 & \cellcolor{cellgreen}1100 & \cellcolor{cellblue}14 & \cellcolor{cellpink}C \\
    13 & \cellcolor{cellgreen}1101 & \cellcolor{cellblue}15 & \cellcolor{cellpink}D \\
    14 & \cellcolor{cellgreen}1110 & \cellcolor{cellblue}16 & \cellcolor{cellpink}E \\
    15 & \cellcolor{cellgreen}1111 & \cellcolor{cellblue}17 & \cellcolor{cellpink}F \\\bottomrule
    \end{tabular}
    \end{table}
    

    \subsection{Convertion examples}    
    \subsubsection*{Binary to Decimal:}

    
      To convert a binary number to decimal, multiply each bit by two raised to the power of its position number, starting from zero on the right.



    \begin{flalign*}
    & \text{Binary:}       & \underline{1} \underline{0} \underline{1} \underline{1} & \\
    & \text{Decimal:}      & 1 \times 2^3 + 0 \times 2^2 + 1 \times 2^1 + 1 \times 2^0 = 8 + 0 + 2 + 1 = 11 &
    \end{flalign*}
    
    \newpage

    \subsubsection*{Decimal to Binary:}

    Let's convert the decimal number \( 25_{10} \) to binary.

    \begin{tabular}{rclr}
    $25 \div 2$ & = & 12 & remainder 1 (LSB) \\
    $12 \div 2$ & = & 6  & remainder 0 \\
    $6 \div 2$  & = & 3  & remainder 0 \\
    $3 \div 2$  & = & 1  & remainder 1 \\
    $1 \div 2$  & = & 0  & remainder 1 (MSB) \\
    \end{tabular}

    \vspace*{10px}
    Thus, the binary representation of \( 25_{10} \) is \( 11001_2 \) (reading the remainders in reverse).
   

    \vspace*{10px}
    \textit{Personal Remark} The trick is always to try to answer the question, what's the biggest power of 2 I need to form the number?. For 157, the biggest power would be $2^{7} = 128$, then $128+64$ is greater than 157, $128 + 32$ is still greater than 157, $128 + 16 = 144$, and so on to obtain : $128+16+8+4+1=157$ which can be written as $2^7 + 2^4 + 2^3 + 2^2 + 2^0 = 157$. Written in binary as $ 10011101_{2} $

    \subsubsection*{Octal to Decimal:}
    
    Each octal digit is converted to decimal by multiplying it by eight raised to the power of its position number, starting from zero on the right.
    
    \begin{flalign*}
    & \text{Octal:}        & \underline{2} \underline{5} \underline{7} & \\
    & \text{Decimal:}      & 2 \times 8^2 + 5 \times 8^1 + 7 \times 8^0 = 128 + 40 + 7 = 175 &
    \end{flalign*}
    \subsubsection*{Decimal to Octal:}
 To convert the decimal number \(93_{10}\) to octal.
    
    \begin{tabular}{rclr}
    $93 \div 8$ & = & 11 & remainder 5 \\
    $11 \div 8$ & = & 1  & remainder 3 \\
    $1 \div 8$  & = & 0  & remainder 1 \\
    \end{tabular}
    
    \vspace*{10px}
    Thus, the octal representation of \(93_{10}\) is \(135_8\) (reading the remainders in reverse).
    
    \subsubsection*{Hexadecimal to Decimal:}

    To convert the hexadecimal number \( \text{1A3}_{16} \) to decimal.

    \begin{flalign*}
    & \text{Hexadecimal:}       & \underline{1} \underline{A} \underline{3} & \\
    & \text{Decimal:}           & 1 \times 16^2 + A \times 16^1 + 3 \times 16^0 & \\
    &                           & 1 \times 256 + 10 \times 16 + 3 \times 1 & \\
    &                           & 256 + 160 + 3 & \\
    &                           & 419 &
    \end{flalign*}

    Here, \( \text{A} \) in hexadecimal corresponds to \( 10 \) in decimal.



    


    
    \subsubsection*{Decimal to Hexadecimal:}
  
    To convert the decimal number \(291_{10}\) to hexadecimal.
    
    \begin{tabular}{rclr}
    $291 \div 16$ & = & 18 & remainder 3 \\
    $18 \div 16$  & = & 1  & remainder 2 \\
    $1 \div 16$   & = & 0  & remainder 1 \\
    \end{tabular}
    
    \vspace*{10px}
    Thus, the hexadecimal representation of \(291_{10}\) is \(123_{16}\) (reading the remainders in reverse).
    


    \section{Octal/Hexadecimal to/from Binary}
    \section*{Bit-Vector Representation Summary}

    \begin{itemize}
      \item Digit-vectors for binary, octal, and hexadecimal systems are represented using bit-vectors. In binary, 0 and 1 are directly represented as 0 and 1.
      \item In systems like octal or hexadecimal, a digit is a bit-vector of length \( k \), where \( k \) is the number of bits needed to represent the base.$$k=\log_{2}(r)$$
      with r the radix of the system (eg. 8 for octal convertion).
      \item For example, the hexadecimal digit \( B \) is represented as the bit-vector 1101 in binary. \textit{We obtain a length 4 bit-vector because the base is 16 and \( \log_{2}(16) = 4 \)}
    \end{itemize}

    \subsubsection*{Binary to Octal:}

    To convert a binary number to octal, group every three binary digits into a single octal digit, because \( k = \log_2 8 = 3 \).
    
    \begin{flalign*}
    & \text{Binary:}       & \underline{010}\,\underline{000}\,\underline{100}\,\underline{110} & \\
    & \text{Octal:}        & 2_8\,0_8\,4_8\,6_8 &
    \end{flalign*}
    
    \subsubsection*{Binary to Hexadecimal:}
    
    To convert a binary number to hexadecimal, group every four binary digits into a single hexadecimal digit, because \( k = \log_2 16 = 4 \).
    
    \begin{flalign*}
    & \text{Binary:}       & \underline{1011}\,\underline{1110}\,\underline{1010}\,\underline{1101} & \\
    & \text{Hexadecimal:}  & B_{16}\,E_{16}\,A_{16}\,D_{16} &
    \end{flalign*}
    


    \subsubsection*{Octal to Hexadecimal:}
    Convert the octal number to binary, then group the binary digits in sets of four and convert each group to its hexadecimal equivalent.
    
    \begin{flalign*}
    & \text{Octal:}        & \underline{2} \underline{5} \underline{7} & \\
    & \text{Binary:}       & 010 \, 101 \, 111 & \text{ (Octal to binary)} \\
    & \text{Binary grouped:} & \underline{0101} \, \underline{0111} & \\
    & \text{Hexadecimal:}  & 5 \, 7 & \text{ (Binary to hexadecimal)}
    \end{flalign*}    

    \section{Representation of
    Signed Integers}
    \subsection{Sign-Magnitude Representation (SM)}

    A signed integer \( x \) is represented by a pair \( (x_s, x_m) \), where \( x_s \) is the \textit{sign} and \( x_m \) is the \textit{magnitude} (positive integer).

    \begin{itemize}
        \item[] The sign (positive, negative) is represented by a the most significant bit (MSB) of the digit vector:
        \begin{itemize}
            \item[] \( 0 \rightarrow \) positive
            \item[] \( 1 \rightarrow \) negative
        \end{itemize}
        \item[] The magnitude can be represented as any positive integer. In a conventional radix-\( r \) system, the range of \( n \)-digit magnitude is:
        \[ 0 \leq x_m \leq r^n - 1 \]
    \end{itemize}

    \begin{itemize}
      \item[-] Examples:
      \begin{align*}
          01010101_2 &= +85_{10} \\
          01111111_2 &= +127_{10} \\
          00000000_2 &= +0_{10} \\
          11010101_2 &= -85_{10} \\
          11111111_2 &= -127_{10} \\
          10000000_2 &= -0_{10} \\
      \end{align*}
  \end{itemize}
  \newpage
  Note: The Sign-and-Magnitude representation is considered a redundant system because both \( 00000000_2 \) and \( 10000000_2 \) represent zero.

  \begin{itemize}
    \item[] \textit{SM} consists of an equal number of positive and negative integers.
    \item[] An \( n \)-bit integer in sign-and-magnitude lies within the range \textit{(because of 0's double representation and that MSB is used for the sign)}: 
    $$ [-\left(2^{n-1} - 1\right), +\left(2^{n-1} - 1\right)] $$ 
    \item[] Main disadvantage of SM: complex digital circuits for arithmetic operations (addition, subtraction, etc.).
\end{itemize}

\section{True-and-Complement (TC)}

\subsection{Mapping}
\begin{itemize}
    \item[] A signed integer \( x \) is represented by a positive integer \( x_R \), \( C \) is a positive integer called the \textit{complementation constant}.
    \[ x_R \equiv x \mod C \]
    
    \item[] For \( |x| < C \), by the definition of the modulo function, we have:
    \[
    x_R = 
    \begin{cases} 
    x & \text{if } x \geq 0 \quad \text{(True form)} \\
    C - |x| = C + x & \text{if } x < 0 \quad \text{(Complement form)}
    \end{cases}
    \]
\end{itemize}

\subsection{Unambiguous Representation}
\begin{itemize}
    \item[] To have an unambiguous representation, the two regions should not overlap, translating to the condition:
    \[ \max |x| < \frac{C}{2} \]
\end{itemize}


\subsection{Converse Mapping}
\begin{itemize}
    \item[] Converse mapping:
    \[
    x = 
    \begin{cases} 
    x_R & \text{if } x_R < \frac{C}{2} \quad \text{(Positive values)} \\
    x_R - C & \text{if } x_R > \frac{C}{2} \quad \text{(Negative values)}
    \end{cases}
    \]
    \item[] When \( x_R = \frac{C}{2} \), it is usually assigned to \( x = -\frac{C}{2} \).
    \item[] Asymmetrical representation simplifies sign detection.

\end{itemize}

\section{Two's Complement System}
This is the True-and-Compelement system with \( C = 2^n \), where \( n \) is the number of bits used to represent the integer.
\begin{itemize}
    \item[] Range is asymmetrical:
    \[ -2^{n-1} \leq x \leq 2^{n-1} - 1\]
    \item[] The representation of zero is unique.
\end{itemize}


\subsection{Sign Detection in Two's Complement System}

\begin{itemize}
  \item[] Since $|x| < C/2$ and assuming the sign is 0 for positive and 1 for negative numbers:
\end{itemize}

\[
\text{sign}(x) = 
\begin{cases} 
0 & \text{if } x_R < C/2 \\
1 & \text{if } x_R \geq C/2
\end{cases}
\]

\begin{itemize}
  \item[] Therefore, the sign is determined from the most-significant bit:
\end{itemize}

\[
\text{sign}(x) = 
\begin{cases} 
0 & \text{if } x_{n-1} = 0 \\
1 & \text{if } x_{n-1} = 1
\end{cases}
\quad \text{equivalent to} \quad \text{sign}(x) = x_{n-1}
\]
\subsection{Mapping from Bit-Vectors to Values}
The value of an integer represented by a bit-vector $b_{n-1}b_{n-2}\ldots b_1b_0$ can be universally expressed as:
\[
\text{Value} = (-2^{n-1} \cdot b_{n-1}) + \sum_{i=0}^{n-2} b_i \cdot 2^i
\]
where $b_{n-1}$ is the MSB (sign bit) and is $0$ for non-negative numbers and $1$ for negative numbers.




\end{document}
